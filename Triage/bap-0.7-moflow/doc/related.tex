\chapter{Related Work}
\label{vine:related}

\paragraph{Other Binary Analysis Platforms.} \bap is designed to 1)
have a formal, well-defined IL, 2) explicitly expose the semantics of
complex assembly in terms of the simpler \bap IL, and 3) be easy to
re-target.  There are several other binary analysis platforms which
may have some similarity to \bap, do not fulfill all requirements.

Phoenix is a program analysis environment developed by Microsoft as
part of their next generation compiler~\cite{phoenix}. One of the
Phoenix tools allows Microsoft compiled code to be raised up to a
register transfer language (RTL).  A RTL is a low-level IR that
resembles an architecture-neutral assembly.

Phoenix differs from \bap in several ways. First, Phoenix can only
lift code produced by a Microsoft compiler. Second, Phoenix requires
debugging information, thus is not a true binary-only analysis
platform.  Third, Phoenix lifts assembly to a low-level IR that does
not expose the semantics of complicated instructions, e.g., register
status flags, as part of the IR~\cite{phoenix:noflags}. Fourth, the
semantics of the lifted IR, as well as the lifting semantics and
goals, are not well specified~\cite{phoenix:noflags}, and thus not
suitable for our research purposes.


The CodeSurfer/x86 platform~\cite{balakrishnan:2005} is a proprietary
platform for analyzing x86 programs. At the core of CodeSurfer/x86 is a
value-set analysis (VSA)~\cite{balakrishnan:2007}.  We have also
implemented VSA in \bap.  CodeSurfer/x86 was not made available for
comparison, thus we do not know whether it supports a well-defined IL,
is re-targetable, or what other comparable aspects it may share with \bap.

\paragraph{Decompilation.} Decompilation is the process of inverting
the compilation of a program back to the original source
language. Generally, the goal of decompilation is to recover a valid
program in the original source language that is semantically
equivalent (though need not be exactly the same) as the original
source program.  

Program analysis is often an important component in the decompilation
process. Cifuentes has proposed using data and control flow analyses
as part of decompilation~\cite{cifuentes:1994}.  Van Emmerik has shown
that analysis on SSA flow graphs is helpful in a variety of tasks such
as reconstructing types and resolving indirect
jumps~\cite{vanemmerik:2007}.  \bap also has SSA;  thus it should
be straight-forward to implement Van Emmerik's algorithms. Mycroft has
proposed type inference for recovering C types during
decompilation~\cite{mycroft:1999}.  Adding type inference is a
possible future direction for \bap.


\paragraph{Binary Instrumentation.} Binary instrumentation is a
technique to insert extra code into a binary that monitors the
instrumented program's behavior
(e.g.,~\cite{srivastava:1994,larus:1995,luk:2005,nethercote:2004:phd,nanda:2006,dynamorio,dyninst}). Instrumentation
is performed by inserting jumps in the original binary code to the
instrumentation code.  The instrumentation code then jumps back to the
original code after executing. The instrumentation code must make sure
that the execution state is the same before and after the jump in
order for the instrumentation to be transparent.

Although many binary instrumentation tools provide a limited amount of
program analysis, the end-goal is instrumentation, not facilitating
analyses.  For example, Pin~\cite{luk:2005} calculates register
liveness information. However, general static analysis is outside the
scope of such tools. For instance, instrumentation tools generally do
not expose the semantics of all instructions such as register status
flag updates.


\paragraph{Other Program Analysis Platforms.} \bap shares many of the same
goals, such as modularity and ease of writing correct analyses, as
other program analysis platforms.  For example,
CIL~\cite{necula:2002:cil} and SUIF~\cite{suif} are both excellent
platforms for analyzing C code.  However, using higher-level program
analysis platforms is inappropriate for binary code because binary
code is fundamentally different. While higher-level languages have
types, functions, pointers, loops, and local variables, assembly has
no types, no functions, one globally addressed memory region, and
goto's. The \bap IL and surrounding platform is designed specifically
to meet the challenges of faithfully analyzing assembly.


