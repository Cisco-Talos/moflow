\section{Concrete Evaluation}

In this tutorial, we will explain how to raise a binary to the BAP IL,
and then evaluate the resulting code inside of BAP.

Let's start with the following simple C program:
\verbatiminput{chap-examples/test.c}

We can compile this source file (named test.c) to a binary using
\cmdline{gcc -static test.c -o test} (note if you are on an x64 machine you may
need to add the \cmdline{-m32} option).  We use the \cmdline{-static} option of
gcc here because the BAP evaluator needs to know what code is located
at each memory location. This means that BAP does not currently
support self-modifying code, or dynamic linking (unless the user
explicitly tells BAP the mapping of memory addresses to binary code).

Once we have a binary, we can then raise it to the BAP IL by executing
\cmdline{toil -bin test -o test.il}. We can now disassemble the binary
to see where main begins and ends -- we will need this to tell BAP
where to start and stop executing. Here is a possible disassembly of
the main function.

\begin{verbatim}
08048267 <main>:
 8048267:       55                      push   %ebp
 8048268:       89 e5                   mov    %esp,%ebp
 804826a:       83 ec 04                sub    $0x4,%esp
 804826d:       c7 04 24 2a 00 00 00    movl   $0x2a,(%esp)
 8048274:       e8 d7 ff ff ff          call   8048250 <g>
 8048279:       c9                      leave  
 804827a:       c3                      ret    
 804827b:       90                      nop
 804827c:       90                      nop
 804827d:       90                      nop
 804827e:       90                      nop
 804827f:       90                      nop
\end{verbatim}

For this sample disassembly, we can begin execution at address
0x8048267.  Likewise, we can halt execution at 0x804827a.  To make BAP
halt execution, we'll manually add a halt command to the lifted IL.
For this example, you could search for the string
\verb!addr 0x804827a! in test.il, which marks the beginning of the ret
instruction in the lifted IL.  Inserting \verb!halt R_EAX:u32!  after
the address and label statements will cause the evaluator to halt and
return the value in eax.  For instance:

\begin{verbatim}
addr 0x804827a @asm "ret    "
label pc_0x804827a
halt R_EAX:u32
\end{verbatim}

Now, issue \cmdline{ileval -il test.il -eval-at mainaddr}, where
mainaddr is the address to start from.  BAP will evaluate the IL
program.  If debugging output is enabled (by setting the
\texttt{BAP\_DEBUG\_MODULES} environment variable to
\texttt{SymbEval}), it will print each statement as it is evaluated.
After halting, BAP will print the return value, which is
eax. Unsurprisingly, the returned value of eax is 42.

Now let's begin execution starting from the call instruction in main
(0x8048250). This will allow us to modify the input to the function
g. The -init-var and -init-mem options allow the user to specify an
initial value for a variable (register) or memory address
respectively.  To replicate what we just did and use an input of 42,
we can use the following the command \cmdline{ileval -il test.il
  -init-mem R\_ESP:u32 42:u32 -eval-at 0x8048274}. Notice that this
begins execution at the call instruction instead of the beginning of
main.  Unsurprisingly, eax has a final value of 42 again.  If we
instead change the memory pointed to by esp to 43, by \cmdline{ileval
  -il test.il -init-mem R\_ESP:u32 43:u32 -eval-at 0x8048274}, eax has
a final value of -1.
